%  -----------------------------------------------------------------------------
%  Author         : Bimalka Piyaruwan Thalagala
%  GitHub         : https://github.com/bimalka98
%  Last Modified  : 2022-05-10
%  -----------------------------------------------------------------------------

\documentclass[a4paper,12pt]{report}%,twocolumn
\input{settings/packages}
\input{settings/page}
%% my macros

%% Text fomats
\newcommand{\tbi}[1]{\textbf{\textit{#1}}}
\renewcommand\bibname{References}

% IEEE naming convention
%\renewcommand{\figurename}{Fig.}

% custom counter for shell scripts
\newcounter{shellcounter}
% \newcounter{myexample}% preamble
\newtcolorbox[
use counter=shellcounter,
number format=\Alph]{shell}[2][]{%
	colback=blue!5!white,
	colframe=blue!75!white,
	fonttitle=\bfseries,
	title= {\sc Shell \thetcbcounter}: #2,#1}


\newcounter{outputcounter}
\newtcolorbox[
use counter=outputcounter,
number format=\Roman]{result}[2][]{%
	colback=green!5!white,
	colframe=green!75!black,
	fonttitle=\bfseries,
	title= {\sc Result box \thetcbcounter}: #2,#1}

\definecolor{lightpink}{rgb}{1.0, 0.71, 0.76}
\renewcommand\bibname{References}
\usepackage{acronym}



\begin{document}
\input{content/title_page}
%---------------------------------------------------------------------------


\chapter*{Preface}
\addcontentsline{toc}{chapter}{Preface}
% A brief account of the Preface
This report was composed as partial fulfillment of the requirements of Module EN3992 - Industrial Training, in the curriculum of B.Sc. in Engineering (Electronic and Telecommunication) at the University of Moratuwa, Sri Lanka. The experience and knowledge that I gained during the six months of my industrial training were used and were the inspiration to create this report.\\

 

\cleardoublepage
%---------------------------------------------------------------------------



\chapter*{Acknowledgment}
\addcontentsline{toc}{chapter}{Acknowledgment}
%Appreciation of those who helped in the internship process

I would like to gratefully acknowledge all of the people who helped me to make this six months of special industrial training period a massive success, starting from the point of applying to a company to the point I left the training organization at the completion of six months.\\

First of all, I would like to express my heartfelt gratitude to Professor Kapila Jayasinghe who was our supervisor for us throughout the six months of the internship period. The advice he provided us with regard to professional engineering practices and ethics was invaluable. In addition to that the directions he provided us to gain the required technical skills required for the allocated project were priceless. I believe the mindset that he built in us, towards working under minimum supervision, will be a massive support for us to thrive in the fast-moving industry.\\

Next, I would like to express my gratitude toward Miss. Laknie Jayasinghe who was the Engineer in charge of us in our training period. The support she provided us to improve our soft skills as well as technical skills as a professional engineers, is highly appreciated. The support she provided when composing the technical documentation of my allocated task by pointing out the mistakes and the areas of improvement was invaluable and must be mentioned at this point. Moreover, her support in validating the project deliveries at the end of the training period is highly appreciated.\\

Last but not least, I would like to express my heartiest gratitude to Mr Janka Kulathunga, the manager of the Lanka Electronics research and development section and Mr Aloka Perera, an electronic engineer in LE Robotics (Pvt.) Ltd for being available for us whenever we needed their support.



%---------------------------------------------------------------------------

\tableofcontents %Three header levels (e.g. 2.7.1) are adequate
\vfill
\begin{center}
	\textbf{\textit{*PDF is clickable}}
\end{center}

%---------------------------------------------------------------------------
%Descriptions of Abbreviations used
\chapter*{List of Abbreviations}
\addcontentsline{toc}{chapter}{List of Abbreviations}
\begin{acronym}
	\acro{oop}[OOP ]{Object Oriented Programming }
	\acro{rpi}[RPi]{Raspberry Pi}
	\acro{cv}[CV]{Computer Vision}
	\acro{rnd}[R\&D]{Research and Development}
\end{acronym}

Use of acronyms:\\

I have use \ac{oop} here. \ac{oop} is very power full. \Ac{rpi} is a single board computer. 


%---------------------------------------------------------------------------
\listoffigures %Figures (including charts) Numbered as per the Chapter
%---------------------------------------------------------------------------
\listoftables %Tables Numbered as per the Chapter
%---------------------------------------------------------------------------




\pagebreak

\chapter{Description of the organization and business, its past, present and	future}

For my special industrial training, I got the opportunity to work as an engineering trainee at LE Robotics (Pvt.) Ltd. located at 100/4, Divulapitiya Road, Minuwangoda, Sri Lanka.It is a local \ac{rnd} facility where they work on industrial articulated robot arms and related technologies. This chapter provides an extensive description of the organization and business as well as information about its past, present and future.

\section{Organization and Business}

\subsection{Introduction}
LE robotics (Pvt.) Ltd. is a local \ac{rnd} facility that has been established by Prof. J.A.K.S. Jayasinghe who is a senior professor in the Department of Electronic and Telecommunication Engineering at the University of Moratuwa in Sri Lanka.\\

LE robotics (Pvt.) Ltd. is the first in the Sri Lankan market to offer fully customisable robotics solutions `Made in Sri Lanka' for various automation needs for an affordable price with expertise based in Sri Lanka. In the year 2005, the company designed and manufactured the first custom robotics solution in their affiliated company, Lanka Electronics (Pvt.) Ltd. Since then, they have been developing various robotics solutions and related technologies in the facility.

\begin{figure}[H]
	\centering
	\fbox{
		\includegraphics[width=0.5\linewidth]{figures/logoler}
	}
	\caption{Logo of the LE Robotics (Pvt.) Ltd.}
	\label{fig:logoler}
\end{figure}

\subsection{Services and Products}

As mentioned previously LE Robotics Pvt. Ltd. provides services and tailor-made products for various automation needs of its customers. When it comes to the services they provide, the following key services can be highlighted.

\begin{enumerate}
	\item Process Analysis to identify the automation needs and  challenges of organizations
	\item Product customization to provide the best tailor-made solution for requirements
	\item Providing local expertise with ease of access
	\item Providing lifetime support for the products
\end{enumerate}

In addition to the services they provide, the following products are manufactured at LE Robotics Pvt. Ltd. \ac{rnd} of the related technologies of those products is a key activity among the day-to-day activities in the facility. 


\begin{enumerate}
	\item \textbf{6 DOF Robots} - Robots with six degrees of freedom
	\begin{itemize}
		\item Robotics solutions with the capability to mimic human arm operations
		\item Offers the flexibility to provide you with fully customized robotic movements to suit your requirement
		\item  Around 1 m reach and 0.1 mm placement precision
	\end{itemize}
	
	\item\textbf{ 4 DOF Robots} - Robots with four degrees of freedom
	\begin{itemize}
		\item Robotics solutions with superior high-speed performance, high rigidity and high accuracy
		\item Array of options with a compact design
		\item Around 500 mm reach and 0.1 mm placement precision
	\end{itemize}
	
	\item \textbf{Custom Made Robots} - Robots designed according to the customer's needs
	\begin{itemize}
		\item Reach and Payload specifications are possible to be customized as per your process requirements
		\item Robotics Solutions to carry out simple pick and place operations
	\end{itemize}
\end{enumerate}





\chapter{Description of familiarization work carried out}

\chapter{Exposure to systems (HSE, Financial, Administration, Logistics, etc.)}

%===========================================================================
\chapter{Project Work}

By title the project that I was assigned, was \textbf{\textit{Machine vision based Real time Motion Planning for an Industrial Articulated Robot Arm}}. In simple words it was a project related to an automatic pick and place machine which can be used to pick objects placed on a conveyor belt and pass them to the next stage of processing.\\

My contribution to that project was to develop the following three aspects of the system. 

\begin{enumerate}[I.]
	\item Development of an object detection framework: 
	\item Development of an application to train an object classification model
	\item Development of an application for camera calibration
\end{enumerate}

Subsequent sections will thoroughly explain the mentioned sub projects that were undertaken by me.

\section{Development of an Object Detection Framework}

The first project that I was assigned as a trainee electronic engineer was related to Computer vision field. An Object Detection Framework was developed to be deployed in an Automatic Pick and Place Machine. The framework is capable of, identifying regions of interests (ROIs), detecting and classifying objects, determining location and orientation of objects with respect to a real world coordinate system for grasping (picking).

%===========================================================================
\chapter{Hands on experiences}

\chapter{Soft Skills Development}

\chapter{SWOT Analysis of the organization and self}

\chapter{Conclusion: Own perspective of areas to be improved (of the whole training process including self)}


\begin{appendices}
	\chapter{Guidelines}
\end{appendices}

\bibliographystyle{plain}
\bibliography{refer}

%---------------------------------------------------------------------------
\end{document}
-
